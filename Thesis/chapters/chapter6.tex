\chapter{Conclusion}\label{conc}
In this last chapter, the thesis has been concluded and also future research steps are presented.
\section{Contributions}
An exact small-signal study of inverters with different control strategies is necessary for inter-area oscillations and stability analyses in power systems. However, the Small-signal modelings of inverters studied in the literature, are limited due to stand-alone consideration of inverter, lack of proper consideration of propagation of small-signal disturbance through power loop, DC-side dynamics, and most importantly, consideration of global and local #dq# frame differences due to power flow which is particularly a property of large power grids. The contribution of the proposed thesis is summarized as follows:

Due to the complexity and non-linearity and a coupled control framework in the $dq$ frame, a systematic method should be used to simplify the inverter equations. In this regard, a matrix formulation of $d$ and $q$ components is proposed, and every block is presented as a $2 \times 2$ matrix. The matrix formulation can handle $dq$- coupled control blocks and also help to formulate linearized forms of non-linear blocks such as power measurement, DC dynamics, droop, and PLL. In addition, the matrix formulation of each block reduces the system equations to algebraic equations with the order of one. This reduction is crucial in exact inverter modeling because a presentable explicit equation can be derived for the output admittance/impedance in small-signal by solving the set of equations. In order to tackle solving a set of equations with a size of $16 \times 16$, a computational symbolic solver (MATLAB) is used. This implementation makes post-processing easier so that each block term in the admittance/impedance model can be readily replaced with its $2 \times 2$ components. After the replacement process, a $2 \times 2$ admittance/impedance is derived. Then each term forms a transfer function that indicates the propagation dynamics of a small-signal perturbance in the inverter. The parametric derivation of exact inverter output admittance/impedance is the contribution of this thesis because it will be a basis for accurate sensitivity analysis of parameters and their importance in system stability. 
 
Six inverter types are chosen as  \gls{GFL}-basic, \gls{GFL}-\gls{PLL}, \gls{GFM}-droop, \gls{GFM}-Virtual Inertia, \gls{GFM}-Power Synchronization Control, and \gls{GFM}-synchronvertor to be studied. The previous approaches only included either GFL or GFM with a droop or virtual inertia control strategy. In addition, the generation of power signal requires both voltages and currents and a low-pass filter formulated in matrix form. In addition, both currents and voltage are outputs of a $dq/abc$ transform which is affected by the angle perturbation. $abc/dq$ transform in the modulation index also is affected by the angle perturbation. The angle perturbation in the GFL is generated via PLL's small-signal block and exists in the literature, whereas the angle perturbation of GFMs is considered to be zero. However, the angle generation in all of the mentioned GFM inverters is generated via feedback from output power which itself is affected by angle perturbation via both voltage and current $abc/dq$ blocks. Consideration of this new loop is a contribution of this work that has a significant effect on the GFM stability.

In the \ref{results}, first, all inverters are compared in terms of eigenvalues, time domain, and frequency domain responses for given parameters, and their results are discussed. Results show considerable complexity in the GFM formulation and stability issues, especially the syncronvertor strategy, compared with the acceptable performance of GFL. Finally, sensitivity analyses are performed for selected parameters to obtain the most effective elements in the low-frequency oscillations. According to the results, one of the significant parameters acting in stability is steady current loop gain. The parameters, such as current loop integral gain and PLL gains, can have an effect on the settling time and stability of #dq# admittance. It is also observed that the #dq# admittance is the first component of an inverter that gets unstable. 

Another highlight of the results is that there is a narrow stability margin for the steady-state angle difference between global and local $dq$ frames caused by the load flow difference of inverters. This observation is analogous to the angle stability of synchronous generators. As a result, the same approach for analysis of critical angle and critical time in synchronous generators can also be theorized for inverter-based grids.




\section{Future research}

This study is suitable for small-signal and low-frequency phenomena such as inter-area oscillations, and considers all block diagrams in an inverter; therefore, the resultant transfer functions may have 16 to 23 poles (eigenvalues). One direction for future research can be reducing the transfer function to a lower order model, probably with a least square curve fitting in the Nichols diagram, to make the analysis of inter-area oscillation in multiple inverter systems easier. Another direction can be extending the study in a large-signal environment and including the current limiter block in the studies. Finally, another extension may consider active power-sharing and oscillatory voltage control methods as inverter control strategies. 

% text of this chapter goes here