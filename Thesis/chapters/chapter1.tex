\chapter{Introduction and Background}

\section{The 100\% Renewable Grids: The Trend and Emerging Challenges}

As a result of the concerns over worldwide climate change and growing demands for electricity, the integration of significant amounts of renewable energy into power systems has gained increasing traction. In Europe, the US, China, and Australia, legislative acts have been enforced to increase renewable energy usage. For example, Denmark aims to achieve 100\% renewable energy supply by 2050, eliminating the dependency on fossil fuels. In the US, some states have mandated carbon-free electricity for their future electricity grids, e.g., California has mandated 100\% renewable energy supply by 2045, while Minnesota has passed laws to move to a fully renewable energy supply by 2050. These trends have led to the integration of massive wind and \gls{PV} plants up to several hundred megawatts to be connected to the power grid in certain regions. \cite{Saeedifard}. 

The primary renewable energy resources used in the power system, which are solar and wind-based, use inverter-based integration due to their intermittent power generation and lack of mechanical power. Therefore, to fulfill the mentioned goals, the inverter-based power generation may be dominant and force synchronous generators to be obsolete. This structural change may cause problems with power system strength, dynamics, stability, and response to a wide variety of phenomena. These emerging problems require modification of power system apparatus, especially inverters, to maintain their reliable performance in terms of low system inertia, fault-ride-through, response to grid phase shift, post fault recovery, inter-area oscillation damping, black start, current limitations during short-circuits, and frequency and voltage support. Therefore, different approaches are taken towards imposing new standards and modifying inverters to make the system compatible with this new structural change.

\section{Significance of the Technical Problem}

Unlike the synchronous generator-based power systems with a well-established theoretical explanation of dynamics, stability, and power system strength, the theoretical analysis of inverter-based AC grids, especially for the large, primary, and high-voltage systems, remains unexamined. The primary prerequisite for developing such theoretical explanations is the exact inverter modeling in the power systems. This modeling should be comprehensive and include all phenomena that can happen in a system. However, to simplify the analysis, the modeling can be divided into pieces where each explains a set of different phenomena. This phenomena can be classified based on different factors such as magnitude and frequency. One significant phenomenon in the power system operation that may require further study in the inverter-based grid is the inter-area oscillation, which can be classified into the small-signal and low-frequency scope. The most crucial prerequisite for studying the inverter-based grid in small-signal and low-frequency is the individual inverter modeling in the small-signal. 

\section{Summary of the Proposed Method}

In this study, an approach has been developed toward modeling the inverter in the small signal analysis, in which the parametric model of the inverter is developed using a matrix formulation and linearization of specific blocks in small-signal by using a computational algebraic equation solver. Most inverter strategies, such as different \gls{GFM}s and \gls{GFL}s, are covered in the modeling.
In order to fully understand the inverter response to the undergoing phenomena, an accurate inverter model is necessary. This model should include large and small signals to cover all the phenomena. However, in this thesis, which is a part of an ongoing research project, only the exact small-signal model for the newest inverter control strategies and different operation points are parametrically derived. This is to first ensure the nonlinear system's stability on the operation point and second to pave the way to the capture and reduction of the inter-area oscillations.

\section{Thesis Structure}

In the rest of the thesis, first, previous inverter modeling approaches are discussed in  \ref{literature}. In \ref{formulation} the matrix form inverter modeling approach is explained, and a small-signal equation for each inverter block is theoretically derived. These blocks are computationally solved in \ref{implementation} and the resultant equivalent impedance/admittance transfer functions are derived and presented. In \ref{results}, first \gls{GFM}s are compared in terms of the number of the poles and frequency response, and then, parameter sensitivity analyses are performed to highlight the importance of specific parameters. Finally, \ref{conc} concludes the thesis and sheds light on future research.