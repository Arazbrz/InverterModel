
\begin{summary}

Due to the concerns about global warming, its critical environmental consequences, and the availability of fossil fuels in the near future, several power systems in the world will aim to become 100 \% renewable-energy-based. This future grid with 100\% inverter-based generation is structurally different in response to various phenomena. Therefore accurate modeling of inverters in this new structure is a crucial task. This thesis introduces a new parametric high-fidelity small-signal model of the \gls{GFM} and \gls{GFL}. Different control strategies for GFM, such as virtual-inertia-based, synchronvertor, power synchronization control, and droop are considered. New control loops are introduced in which the measured signals are related to the angle generation block in \gls{GFM}s. To obtain the comprehensive model, first, nonlinear blocks (power measurement, DC-side dynamics, $dq/abc$ frame transforms, \gls{PLL}, and GFL droop) are linearized, and inverter block diagrams are converted to matrix form subsequently. Then, a set of first-order algebraic equations are formed based on the matrix equations and computationally solved in MATLAB. After derivation of output admittance in matrix form, derived block components are replaced to form four transfer functions that define the dynamics of the inverter. The results show the significant dependency of inverter stability on a steady-state angle obtained via load flow and current loop proportional gain. Therefore a stability margin for steady-state angle is derived, which is analogous to the angle stability of synchronous generators. The synchronvertor controller shows a very complex transfer function with the most unstable and challenging behavior among the rest of the inverters.


\textbf{Keywords:}- Grid Forming Inverters - Inverter Modeling - Inter-Area Oscillations

\end{summary}